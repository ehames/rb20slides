\documentclass[compress]{beamer}

\mode<presentation>
{
  \usetheme{Boadilla}
  \setbeamercovered{invisible} % covered items don"t show up
}

\usepackage[spanish]{babel}
\usepackage[utf8]{inputenc}
\usepackage[T1]{fontenc}
\usepackage{listings}
\usepackage{color}
\usepackage{url}

\definecolor{darkgreen}{rgb}{0,0.4,0}

\title[Lo nuevo en Ruby 2.0]
{\textsc{Ruby~2.0}: ¿Qué hay de nuevo, viejo?}

\author[Edgardo~E.~Hames] {Edgardo~E.~Hames}
\institute{ehames}

\date{5 de abril de 2013}

\subject{Computer Science}
\keywords{Ruby 2.0, programming languages, programming}

\beamerdefaultoverlayspecification{<+->}

\begin{document}
\lstset{
	language=Ruby,
%	frame=single,
%	backgroundcolor=\color{lightgray},
	tabsize=2,
%	numbers=left,
	numberstyle=\tiny,
	numbersep=5pt,
	breaklines=true,
	showstringspaces=false,
	basicstyle=\footnotesize,
	identifierstyle=\color{blue},
	keywordstyle=\bfseries,
	commentstyle=\color{darkgreen},
	stringstyle=\color{red},
	morekeywords={refine, using, \%, i, I, w},
	alsodigit={digit,_}
}

\section{}

\begin{frame}
  \titlepage{}
\end{frame}

\section{Intro}


\begin{frame}
	\begin{center}
		Ruby 1.8 tiene los días contados.
	\end{center}
\end{frame}

\begin{frame}
	\begin{center}
		Múdense a Ruby 2.0 ya!
	\end{center}
\end{frame}

\section{Paráms}

\begin{frame}
	\begin{center}
		Parámetros con nombre\\
		(\emph{keyword arguments})
	\end{center}
\end{frame}

\begin{frame}
\frametitle{Ruby~2.0: parámetros con nombres}
	\begin{itemize}
		\item Podemos asociar nombres a parámetros de un método
		\item Más descriptivos y fáciles de recordar
		\item El orden no es relevante
	\end{itemize}
\end{frame}

\begin{frame}[fragile]
\frametitle{Parámetros con nombres: ejemplo}
	\begin{lstlisting}
	def wrap(string, before: "<", after: ">")
	  "#{before}#{string}#{after}"
	end
	\end{lstlisting}

	\lstset{language=bash}
	\begin{lstlisting}
	wrap("hello")                        #=> "<hello>"
	wrap("hello", after:"]", before:"[") #=> "[hello]"
	\end{lstlisting}

\end{frame}


\begin{frame}[fragile]
\frametitle{Ruby~1.8: simulados usando un hash}
	Usado en Rails y otros DSLs
	\begin{lstlisting}
	some_method :foo => "bar"
	\end{lstlisting}
	
	Claves desconocidas pasaban inadvertidas
	\begin{lstlisting}
	some_method :unknown => "never used"
	\end{lstlisting}
\end{frame}


\begin{frame}[fragile]
\frametitle{Ruby~2.0: mejor que un simple hash}
	\begin{lstlisting}
	begin
	  wrap("hello", unknown: "]")
	rescue ArgumentError => e
	  e.message
	end
	\end{lstlisting}
\end{frame}


\section{Refin}

\begin{frame}
	\begin{center}
		Refinamientos\\
		(\emph{refinements})
	\end{center}
\end{frame}


\begin{frame}[fragile]
\frametitle{Ruby~1.8: las clases pueden modificarse}
	Permite hacer \emph{monkey patching}.
	\begin{lstlisting}
	class String
	  def reverse
	    # hacer otra cosa
	    self.capitalize!
	  end
	end
	\end{lstlisting}
	Contra: ensucia el entorno de las clases clientes.
\end{frame}

\begin{frame}[fragile]
\frametitle{Ruby~2.0: tienen visibilidad acotada}
	\begin{lstlisting}
	module IncompatibleStrings
	  refine String do
	    def reverse
	      # hacer otra cosa
	      self.capitalize!
	    end
	  end
	end
	\end{lstlisting}
	
	\begin{lstlisting}
	"abc".reverse                               #=> "cba"
	using IncompatibleStrings; "abc".reverse    #=> "Abc"
	\end{lstlisting}
\end{frame}


\section{Iter}

\begin{frame}
	\begin{center}
		Iteración perezosa\\
		(\emph{lazy iteration})
	\end{center}
\end{frame}

\begin{frame}[fragile]
\frametitle{Ruby 1.8: iteraciones sobre enumerados}
	\begin{lstlisting}
	(1..1_000_000).map{|i| 2*i}.first      #=> 2
	\end{lstlisting}
\end{frame}


\begin{frame}[fragile]
\frametitle{Ruby 2.0: iteración perezosa}
	La iteración puede ser postergada usando \verb+#lazy+

	\begin{lstlisting}
	(1..1_000_000).lazy.map{|i| 2*i}.first #=> 2
	\end{lstlisting}
\end{frame}



\section{Lit}

\begin{frame}
	\begin{center}
	Nuevos literales
	\end{center}
\end{frame}

\begin{frame}[fragile]
\frametitle{Literal de hash}
	\begin{lstlisting}
	{:on => 1, :off => 2}
	\end{lstlisting}
	
	se puede escribir como
	
	\begin{lstlisting}
	{on: 1, off: 2}
	\end{lstlisting}
\end{frame}


\begin{frame}[fragile]
\frametitle{Arreglos de símbolos \%i y \%I}
	Podemos crear arreglos constantes de símbolos
	\begin{lstlisting}
	% i{on off}   #=> [:on, :off]
	\end{lstlisting}
	
	incluso con interpolación de variables
	\begin{lstlisting}
	s1, s2 = % w{on off}
	% I{#{s1} #{s2}}   #=> [:on, :off]
	\end{lstlisting}
\end{frame}

\section{Otras}

\begin{frame}
	\begin{center}
	Otras mejoras
	\end{center}
\end{frame}

\begin{frame}
	\begin{itemize}
		\item Convención \texttt{to\_h} para convertir a hash
		\item Nuevo motor de expresiones regulares
		\item Module\#prepend para extender clases
		\item UTF-8 por defecto
		\item DTrace/TracePoint para depurar
		\item Optimizaciones en la VM
			\begin{itemize}
				\item menor uso de memoria gracias a bitmap GC
				\item \texttt{require}
				\item operaciones de punto flotante
				\item llamadas a métodos
			\end{itemize}
	\end{itemize}
\end{frame}


\section{}

\begin{frame}{}
  \begin{center}
  {\Huge ¡gracias por venir!}
  \end{center}
\end{frame}

\begin{frame}[fragile]
\frametitle{Referencias}
  \begin{itemize}
  \item \href{http://goo.gl/WMq2a}{Ruby 2.0.0-p0 is released}
  \item \href{http://goo.gl/9E0Tk}{NEWS for Ruby 2.0.0}
  \item \href{http://goo.gl/IAAOF}{Ruby 2.0 (en)}
  \item \href{http://goo.gl/W6pGK}{Ruby 2.0.0 in detail}
  \item \href{http://goo.gl/oNC3m}{Ruby 2.0.0 by example}
  \item \href{http://goo.gl/fbsXD}{Why you should be excited about garbage collection in Ruby 2.0}
  \end{itemize}
\end{frame}


\end{document}
